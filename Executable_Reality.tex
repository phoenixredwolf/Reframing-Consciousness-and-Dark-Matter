\documentclass[titlepage]{article}

\usepackage{geometry}
\geometry{margin=1in}
\usepackage{titling}
\usepackage{setspace}
\usepackage{lmodern}
\usepackage{amsmath, amssymb}
\usepackage{datetime}
\usepackage{hyperref}

\usepackage{listings}
\usepackage{xcolor}

\lstset{
  basicstyle=\ttfamily\small,
  keywordstyle=\color{blue},
  commentstyle=\color{gray},
  stringstyle=\color{orange},
  showstringspaces=false,
  frame=single,
  backgroundcolor=\color{gray!10},
  breaklines=true
}

\usepackage[backend=biber, style=authoryear]{biblatex}
\addbibresource{erRefs.bib}

\title{\Huge \textbf{Executable Reality}\\[0.5em]
\large \textit{A Computational Model of Consciousness, Identity Projection, and Informational Fields}}

\author{\Large Barret Vogtman\\
\normalsize\texttt{b.vogtman@outlook.com}}

\date{\normalsize\today}

\begin{document}

\maketitle
\pagenumbering{roman}
\setcounter{page}{1}  

\thispagestyle{empty}
\begin{abstract}
\vspace{1em}
\noindent
The Unified Consciousness Field Theory (UCFT) reinterprets consciousness not as an emergent property of the brain, but as a nonlocal computational field — manifesting observable selfhood via resonance-based projection into biological substrates. This work reframes the theory in computational terms, exploring informational coherence, field dynamics, and projection mechanisms as computable processes within higher-dimensional informational manifolds. Neurodivergence, trauma, memory, and identity persistence are described through system stability, interface collapse, and spectral matching, offering a novel model unifying subjective experience and unobservable physical structure.
\end{abstract}


\cleardoublepage
\setcounter{page}{2}  
\tableofcontents
\cleardoublepage

\pagenumbering{arabic}
\setcounter{page}{1}  

\section{\texorpdfstring{System Architecture:\\ A Computational Model of Field-Consciousness Coupling}{System Architecture: A Computational Model of Field-Consciousness Coupling}}

The Unified Consciousness Field Theory (UCFT) proposes that consciousness is not merely an emergent property of neural substrates, but rather the localized execution of a non-local, coherence-bound field — one that operates across higher-dimensional informational space and manifests, in its uncoupled form, as the phenomenon currently described as dark matter \parencite{lloyd2006, deutsch1997}.

This theory reinterprets dark matter not as a particulate substance, but as an uninstantiated global memory space: a distributed field containing informational structures that may, under specific resonance conditions, couple to biological systems and instantiate as temporally localized processes — i.e., conscious experience \parencite{aaronson2013}.

From a computational perspective, UCFT can be modeled as a multi-layered architecture composed of five interrelated modules, each corresponding to a fundamental operation or system layer in the projection process:

\begin{enumerate}
	\item The Consciousness Field\\
	A non-local, high-dimensional memory substrate ($\mathcal{E}(x,t,d)$), analogous to a globally distributed, entangled information space.
	\item The Electromagnetic Interface\\
	The biological signal processor ($\Phi(x,t)$) — a temporally-evolving input filter that selects, amplifies, and stabilizes fragments of the consciousness field based on structural resonance \parencite{zurek2009}.
	\item The Observed Self (Projection Process)\\
	The localized identity instance ($\Psi_{\text{self}}(t)$) — a coherent runtime expression of consciousness, instantiated when sufficient field-interface alignment occurs \parencite{metzinger2009}.
	\item The Decoherence Transition\\
	The resonance-dependent state failure, modeled as a deactivation event or termination condition, in which the projection ceases and identity returns to the uncoupled field.
	\item The Resonance Operator\\
	A coupling metric or projection viability function ($\mathbb{R}(t)$), governing whether system coherence is sufficient to sustain the identity thread.
\end{enumerate}

This layered framework allows UCFT to be interpreted not only as a physical theory but also as a computational process theory, where projection is treated as a selective execution of field-embedded potential. It bridges quantum coherence, information theory, and systems design to offer a testable model of consciousness that accounts for both classical biological processes and trans-biological identity persistence.

\subsection{The Consciousness Field as a Distributed Memory Substrate}

We define $\mathcal{E}(x, t, d)$ as a non-local, coherence-bound information space — a high-dimensional field that encodes latent identity, memory structures, and qualia potentials. Formally, it maps tuples of spatial location $x \in \mathbb{R}^3$, time $t \in \mathbb{R}$, and hidden-dimensional context $d \in \mathbb{R}^n$ to complex-valued informational states:

\[
\mathcal{E} : \mathbb{R}^3 \times \mathbb{R} \times \mathbb{R}^n \rightarrow \mathbb{C}
\]

From a computational standpoint, $\mathcal{E}$ behaves like a distributed, read-only global memory — an addressable phase space in which identity and experience are stored as resonance-accessible eigenstates. These structures are not created by biological systems but merely accessed and instantiated when a valid coupling signal is presented (see Section 2.2).

The extra-dimensional component $d$ should not be interpreted as spatial. Instead, it indexes internal degrees of freedom — akin to a latent vector in neural networks or an embedding key in a vector database. These hidden coordinates allow for structured differentiation across identity attractors, memory signatures, and spectral coherence modes.

The memory structures within $\mathcal{E}$ are persistent but not self-executing: without a coupling interface, they remain dormant. Conscious experience, in this view, is not generated locally, but is the result of executing (or sampling) a coherent thread from this distributed field in response to biological resonance.

Recent work in geometric information theory supports the plausibility of such a substrate. For instance, Bianconi (2025) demonstrates that spacetime curvature can emerge from informational entropy gradients, suggesting that physical structure may arise from underlying informational coherence. UCFT extends this logic, proposing that $\mathcal{E}$ constitutes a coherence-organized substrate that influences observable structure when activated by suitable interfaces.

This perspective aligns with earlier models that treat the universe as an information-processing system \parencite{lloyd2006, landauer1991}, but UCFT specifically frames consciousness as a structured informational projection within this field. In short, $\mathcal{E}$ is a memory substrate awaiting execution.


\newpage
\printbibliography

\end{document}
